\section{Ghost cells}
In order to reconstruct the states to the left and right of the first and last interface, I needed to use imaginary ghost cells in which I fill the values of the conserved quantities as specified by the boundary conditions each iteration. I then could use these values of the ghost cells in order to compute the left and right states during reconstruction. For first order in space, I only needed one ghost cell, whereas for second order in space, when I did the TVD minmod reconstruction, I needed two ghost cells on either side.
\subsection{First order in Space}
For first order in space, we simple use the left and right states on either side of each boundary to compute the numerical flux across the boundary. Two first order, we do a simple godunov scheme without any reconstruction, and we just use the time averaged values across each of the left and right states as the left and right states for the local riemann problem. When this is the case, we only need one ghost cell on either side of the leftmost and rightmost boundary. 
\subsection{Second Order in Space}
For second order in space, we do a minmod reconstruction of the values of each of the conerved variables or primitive variables on either side of each of the interfaces. Using the generalized minmod scheme, to reconstruct the values of the left and right states for the local riemann problem at each boundary, we need two states on either side of each boundary in order to reconstruct the states at the boundary. For this reason we need two ghost cells on either side to reconstruct the values of the state variables close to the boundaries. 
