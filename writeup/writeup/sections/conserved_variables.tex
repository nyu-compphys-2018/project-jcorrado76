\section{Conserved Variables}

\subsection{Nonrelativistic Case}
The euler equations are given by:
$$\frac{\partial \bm{U}}{\partial t}+\frac{\partial \bm{F}}{\partial x}=0$$
In this case, the conserved variables are given by:
$$U=(\rho , \rho v , E)^T$$
Where $\rho$ is the fluid density, $v$ is the velocity, $P$ is the pressure, and $E$ is the total energy density. The expression for the total energy density is $E=\rho \epsilon + \frac{1}{2} \rho v^2$ where $\epsilon$ is the specific internal energy. These equations are closed by the equation of state for an ideal gas, given by:
$$p=(\gamma-1)\rho \epsilon$$

\subsection{Relativistic Case}
In the relativistic case, the conserved quantities are slightly different, given by $U=(D,S,\tau)^T$. In this expression, $D$ is the rest mass density, $S$ is the momentum density, and $\tau$ is the energy density, all as measured in the laboratory frame. The relations relating these conserved variables back to the primitives are:
$$D= \rho W$$
$$S = \rho h W^2 v$$
$$\tau = \rho h W^2 - p - \rho W $$
$$W^2 = \frac{1}{1-v^2}$$
For these conversions, $h=1+\epsilon +p / \rho$ is the relativistic specific enthalpy, and $\epsilon$ is the specific internal energy. 
